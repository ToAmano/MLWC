 \begin{abstract}
 Recent advance in the crystal structure prediction and high pressure technology has revealed
 We theoretically address the electronic states emerging from the combination of atomic and molecular orbitals spatially overlapping with each other. Due to the orbital overlap, some eigenstates formed by their superposition must become spatially fragmented, resulting in high kinetic energy. With this effect, the Hilbert space spanned by these orbitals extends to very high energy regime and, we point out that, at any threshold distances between the orbitals, its upper bound diverges to infinity. This means that the low-energy electronic states can not be well described by the molecular orbitals for isolated systems. Anomalous chemical bonds seen under strong compression are generally thought to be the consequence of the above fact; directional bonding and clathrate formation of hydrogen, interstitial electronic occupation, etc.
 \end{abstract}