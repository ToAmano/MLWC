\documentclass[twocolumn,showpacs,prb,amsfonts,amsmath,amssymb,floatfix,groupedaddress]{ltjsarticle} %revtex4-1}
\usepackage{luatexja} % ltjclasses, ltjsclasses を使うときはこの行不要%
\usepackage[ipa]{luatexja-preset}
\usepackage{color}
\usepackage{hhline}
\usepackage{mathrsfs}
\usepackage{graphicx}
\usepackage{dcolumn}
\usepackage{bm}% bold math
\usepackage{multirow}
\usepackage{booktabs}
\usepackage{afterpage}
\usepackage{amsmath,amssymb}
\usepackage{ulem}


\newcommand{\red}[1]{\textcolor{red}{#1}}
\newcommand{\blue}[1]{\textcolor{blue}{#1}}
\arraycolsep=0.0em
\setlength{\abovecaptionskip}{0mm}
\setlength{\belowcaptionskip}{0mm}
%\setlength{\MidlineHeight}{2pt}


\newcommand{\ph}{\phantom{0}}

\renewcommand{\topfraction}{1.0}
\renewcommand{\bottomfraction}{1.0}
\renewcommand{\dbltopfraction}{1.0}
\renewcommand{\textfraction}{0.1}
\renewcommand{\floatpagefraction}{0.9}
\renewcommand{\dblfloatpagefraction}{0.9}
\begin{document}


%\title{Electronic theory for anomalous chemistry under compression}
\title{格子動力学と誘電特性~ Born and Fuangの教科書から}
\author{Tomohito Amano$^{1}$}
%\thanks{amano@cms.phys.s.u-tokyo.ac.jp}
%\affiliation{$^1$Department of Physics, The University of Tokyo, Hongo, Bunkyo-ku, Tokyo 113-0033, Japan}

\date{\today}
 \begin{abstract}
 Recent advance in the crystal structure prediction and high pressure technology has revealed
 We theoretically address the electronic states emerging from the combination of atomic and molecular orbitals spatially overlapping with each other. Due to the orbital overlap, some eigenstates formed by their superposition must become spatially fragmented, resulting in high kinetic energy. With this effect, the Hilbert space spanned by these orbitals extends to very high energy regime and, we point out that, at any threshold distances between the orbitals, its upper bound diverges to infinity. This means that the low-energy electronic states can not be well described by the molecular orbitals for isolated systems. Anomalous chemical bonds seen under strong compression are generally thought to be the consequence of the above fact; directional bonding and clathrate formation of hydrogen, interstitial electronic occupation, etc.
 \end{abstract}
\maketitle


\section{Introduction}


\section{Classical lattice theory}

\subsection{lattice coordinate}


\subsection{Classical lattice theory}
Bornの教科書に従って,各種物理量の基準座標での展開を考える.


\subsection{quantum lattice theory}
非調和効果を$2$次摂動まで考える場合,自己エネルギーのダイヤグラムは$5$つ存在する.


\subsection{self-consistent phonon}
 非調和効果まで含んだフォノンの振動数はグリーン関数$G_q$の極に対応する.すなわち$G_p^{-1}$のゼロ点なので,ダイソン方程式
 \begin{align}
  G_q^{-1}(\omega)= \left[ G_q^{0}(\omega)\right]^{-1}-\Sigma_{q}(\omega)
 \end{align}
 から,
 \begin{align}
  V
 \end{align}
 である.ここで自己エネルギーに対して近似を行う.


\section{Diagramatic approach to Dielectric function}
\subsection{Linear Response Theory}
\subsection{Dielectric Function}


%\begin{acknowledgments}
%This work was supported by (hydrogenomics) and JSPS KAKENHI Grant Numbers 20K20895 from Japan Society for the Promotion of Science (JSPS). 
%\end{acknowledgments}

\bibliography{reference}

\end{document}



